\chapter{Background}

\section{State of the Art}

\subsection{A very short history of Machine Learning}
``You have to know the past to understand the present.'' - Carl Sagan

The early beginnings of Machine Learning come not after the first electronic 
computers or after the first computer programs, as many may believe so. The 
fundamentals of this field took shape centuries ago with the discovery of the 
so-called Conditional Probability theories. {\bf Bayes Theorem}, named after Thomas 
Bayes who first proposed a mathematical model for infering probabilities of 
conditioned events, and further developed by Pierre-Simon Laplace in his essay 
\footnote{Théorie Analytique des Probabilités} in 1812, can be seen as one of the first models 
that can ``learn'' from given data and predict events based on correlated past events.
Another old discovery that is the basis of today's regression models (e.g.: 
Linear Regression) is the ``least squares method'', credited to Carl Gauss, but 
first published by Adrien-Marie Legendre in 1805. This method was first applied in 
astronomy and allowed explorers to navigate oceans by aproximating the movement of 
celestial bodies.

Later on, in 1950, Alan Turing proposed in his paper\footnote{Turing, A.M. (1950). 
Computing machinery and intelligence. Mind, 59, 433-460.} a {\it learning machine} 
that is able to learn and become intelligent and do well in the {\bf Imitation Game} 
(now generally called the {\bf Turing Test}). 

After this, the discovery of the Percetron and the {\bf Neural Networks} around 1960s drew 
some attention in the field, but their current limitations had put Machine Learning on an 
impeding state for almost 10 years. It was only with the invention of the 
backpropagation algorithm in 1974 by Paul Werbos and the demonstration of its 
generalization by Geoffrey Hinton in 1986, that allowed it to be applied in 
multi-layered artificial neural networks. This also gave birth to a new sub-field 
of Machine Learning that today is called {\bf Deep Learning.}

Along with the research in neural networks, some other models that were developed in that 
period are worth to mention. The most important ones are Support Vector Machines and kernels 
(models used for data classification and regression, that can be more time-efficient 
than neural networks are and provide good performance from a data perspective) and Decision Trees. 
The latter, in combination with Ensemble Methods helped researchers invent models like 
{\bf Random Forests} and {\bf Adaptive Boosting} that are now state-of-the-art 
algorithms for tree models used for a lot of tasks.

\subsection{Current interests in the field}

Coming back to Deep Learning, which is today's main subject of interest of the 
Machine Learning community, it is a general approach that combines several state-of-the-art 
models of Machine Learning to solve problems such as image classification, AI for 
computer games, natural language, etc. Its constituents include neural networks with many 
hidden layers, convolutional networks, deep belief networks and recurrent networks. 
Also, the Q-Learning algorithm\footnote{Watkins, C.J.C.H. (1989). Learning from Delayed Rewards. 
PhD thesis, Cambridge University, Cambridge, England} and the Monte-Carlo search used 
in combination with convolutional networks allowed researchers to build semi-supervised 
learning programs that could learn to play computer games by themselves\footnote
{https://www.cs.toronto.edu/~vmnih/docs/dqn.pdf} or beat professional human players at 
games, such as Go (Google AlphaGo's program first beat Lee Sedol in October 2015). 

\subsection{Trends in educational learning}

All advances in the Machine Learning field also conducted in an incresing interest in 
educational learning and assessment. With the name of {\bf Educational Data Mining}, 
this newly emerging discipline deals with studying machine learning and data mining 
models in order to gain important knowledge about the structure of an educational 
system data (final grades, performance indicators, course drop-outs). Educational 
data can be taken from schools and universities (the classical way and also, the way 
that this thesis explores), online courses (such as MOOCs\footnote{Massive Open 
Online Courses}), or even collaborative learning.

With the use of Machine Learning techniques, student can better decide on what courses 
they could take (based on past related grades), whether they have a chance or not to 
pass an exam before taking it and what indicators are relevant to their final asessment. 
The course department also benefits from the ``learned'' data because it helps them 
to better plan the structure of their courses, whether they are on-line or taken at 
the university.

\section{Related Work}

This part of the chapter will focus on the work on other people about the study 
on student performance prediction and analysis. Some of their research is similar 
to that of this thesis, and some others treat only related issues. 

(Mehdi Sajjadi et al., 2016)\cite{bibl_1} did some work on approximating final grades 
of students in a course on algorithms. In the grading process they used the peer 
grading method\footnote{A process in which students grade work of other students 
based on a given guideline}, and then applied Machine Learning (both supervised and 
usupervised) to aggregate those grades into a final grade. Their results were 
not so good compared to the simple method of just using the mean of all peer grades 
per an assessment as the final grade. 

(Siddharth Reddy et al., 2015)\cite{bibl_2} worked on developing a representational 
model of combined students and educational content (assessments and lessons). 
This representation is actually a semantic space\footnote{Semantic similarity 
between objects represented as a kind of ``metric'' in space} and it can be used 
to study the relation between course content and students. Several conclusions can 
be drawn from these representations, such as: probability of passing an assessment 
or course and knowledge gained from completing a lesson. This article aimed mostly 
at MOOCs platforms, like Coursera, EdX and Khan Academy, and the model described 
was tested on synthetic student data and also, on real data from Knewton.
Their model's results can be used to personalize the learning process of a course 
for each student in order to maximize the educational performance. Also, this 
model successfully predicted assessment results.

(Michael Wu, 2015)\cite{bibl_2_1} wrote an interesting Master Thesis in which 
describes a Machine Learning Model that simulates MOOC data. Working with data 
gathered from EdX, the model once trained, can be able to synthesize student data.
The model was trained to learn about student types, habits and difficulty of course 
materials. One of the main results of the thesis was being able to classify 
students in 20 important clusters.

(Saeed Hosseini Teshnizi and Sayyed Mohhamad Taghi Ayatollahi, 2015)\cite{bibl_3} 
did a comparasion between Logistic Regression and ANNs\footnote{Artificial 
Neural Networks} on a dataset composed of 275 undergraduate students and 16 
student characteristics (e.g.: age, gender, parent education, employment status, 
place of residence, etc.) in order to predict academic failure. They concluded that 
the neural network models had a better accuracy than Logistic Regression (84.3\% 
versus 77.5\%). They tested 9 ANNs from which the one with 15 neurons in the 
hidded layer provided the best results, so ANNs methods were appropiate to be used 
in their problem. 

Other references\cite{bibl_4},\cite{bibl_5},\cite{bibl_6} also treat prediction 
of student academic performance mostly with neural networks and provide good 
accuracy with this model (the accuracy is also dependent of the structure of 
the dataset, number of examples, number of characteristics and noise in the 
data).

(Emaan Abdul Majeed and Khurum Nazir Junejo)\cite{bibl_7} had some great results 
using Machine Learning models for predicting student's performance. They claim 
that they were capable of predicting the final grade with an accuracy of 96\%. 
With a final number of 2500 student records and about 10 attributes for each 
record, they managed to predict the value of the final Grade attribute (which 
is a Class Variable that can have 6 values: A, B+, B, C+, C, Fail). 
Four classifier models were used in their study and we can see in Figure 
\ref{fig.figure1}\cite{bibl_7} their performance: 

\insfigshw{figure1.png}%
    {Classifiers accuracy}%
    {Classifiers accuracy}%
    {fig.figure1}{0.8}
