\chapter{Implementation}

This chapter describes the design and implementation of the models expanded in 
Chapter 3, from a more technical perspective. The structure of the dataset and 
also short code listings will be used to show the logic of the implementation. 

In the first section of this chapter, 
we are going to see the approach taken to gather the dataset and parse it to our 
{\it Python} program. The user can choose which subset of the features it wants 
to use in order to apply the ML models and get the desired results.

Next, we are going to focus on the {\it scikit-learn Python} library, which is 
an open-source Machine Learning library with many implemented features. Here, 
we will show how we used our chosen models with the help of this library.  

At the end of the chapter we briefly discuss how the whole implementation was 
tested with data that did not take part in the training process.

\section{Parsing the students data}

All the data used in this project was taken from the course catalog and from 
the {\bf Moodle} platform. The gathered data was put into \texttt{.csv} files, 
one for the semester grades and one for the logs taken from the on-line course 
platform. 

The first line of each \texttt{.csv} file holds the names for the values found in 
the next line. Each further line contains the key string of the example along 
with the feature values and label values that were measured for it. An example 
taken from the {\it grades} file is given in the following listing: 

\lstinputlisting[language=Python]{test.py}

The second file completes the input features of the first one, with the mention that the 
final \texttt{.csv} file was build by a {\it python} script, because we had to 
aggregate all the measurements present in multiple \texttt{.csv} files into 
a single file, with the same structure and order as the one showed above. So, the 
second file contains the course platform log features that were described in table 
\ref{table_logs}.

For loading the dataset into memory, we used a source file called 
\texttt{data\_loader.py}. Here, all the data from both \texttt{.csv} file is loaded 
into a \texttt{python dict()} data structure. Then, using the options defined visible in 
the code listing example \ref{parse_ex}, we can choose which subset of our total extracted 
features are we going to use. 

\lstinputlisting[label=parse_ex,caption=Parsing the data,language=Python]{options.py}

In the above code, the desired dataset is loaded into a \texttt{DatasetContainer} class 
with the \texttt{get\_data(opt, dataset)} function. The \texttt{load\_data} function 
reads the content of the \texttt{.csv} files and puts it into a dictionary. After 
this, the current dataset is preprocessed (scaled and added polynomial features 
to it, if desired) by the \texttt{preprocess\_data} function. Also, in this function, 
we implemented the PCA analysis of the dataset. 

\section{scikit-learn}

The {\it scikit-learn} library is a free machine learning library implemented in 
the {\it Python} programming language, with some base algorithms written in {\it C}, 
for performance considerations. It is simple to use and has a wide range of 
tools for data analysis and machine learning projects. The backend of the library 
uses {\bf Numpy, SciPy and matplotlib} packages developed for scientific purposes. 

The library provides support for all the machine learning subfields, from 
classification to regression and clustering problems. Throughout the development 
of this project, we experimented with machine learning algorithms implemented 
in this library and also, tools for preprocessing and visualizing the data and 
the results of our models. 

\subsubsection{General Usage}

For preprocessing we used the \texttt{StandardScaler()} object to scale our raw 
data, with the goal of having zero mean and unit variance of the data. Also, for generating 
our polynomial features based on the initial features, 
\texttt{PolynomialFeatures()} object from the \texttt{scikit-learn} library was used. 

The {\it PCA} algorithm is also implemented in the library, so we used it to 
generate our two- and three-dimensional plots exposed in Chapter 3. For our 
classifier models evaluation, we needed to calculate and visualize the 
{\it confusion matrix} which evaluates the performance of the classifier 
(more details on this will be given in the following chapter). We used the 
\texttt{confusion\_matrix} function from the library to calculate the matrix. 

The {\it scikit-learn} library also has implementations for many evaluation 
metrics of the models, which can be found in the \texttt{sklearn.metrics} 
module. We used those metrics in our evaluation process and also for comparing the 
models.

\subsubsection{Machine Learning Algorithms}

The models discussed in section \ref{models}, i.e. the {\bf Linear Regression, 
Perceptron, Neural Network and Random Forest} models were all used from the 
{\it scikit-learn} library. For the neural networks models (the classifier and 
the regressor), we had to use a {\it development} version of the library, since 
they are not currently implemented in the stable version. 

In the code listing \ref{nn_clf}, a neural network binary classifier is used 
to train our dataset. The \texttt{model\_eval} function evaluates the model 
based on training and testing data performance metrics. 

\lstinputlisting[label=nn_clf,caption=Neural Network Classifier,language=Python,basicstyle=\footnotesize]{nn_clf.py}

Using the random forest classfier only requires changing the \texttt{clf} 
variable in the code listed above to: \texttt{clf = RandomForestClassifier(n\_estimators=10, 
max\_features='sqrt')} and the rest of the code can remain unchanged (the library 
gives us a standard interface for the training algorithms, no matter which one 
is used). 

If we want to see the feature ranking computed by the random forest model, we 
need to use the \texttt{clf.feature\_importances\_} attribute of the \texttt{clf} 
classifier object. Having this array of values - a probability distribution over 
the feature set, it is easy to plot the feature importances and their names on 
a graph, along with their standard deviations. The \ref{rf} code listing shows 
how we got the feature ranks from our trained model in order to plot them. 

\lstinputlisting[label=rf,caption=Plot Feature Ranks,language=Python,basicstyle=\footnotesize]{rf.py}

\section{Testing}

The testing process was done after we trained each of the models. From the 
\ref{nn_clf} code listing it can be seen that we have kept a subset of our entire 
dataset for the testing purposes (actually 20\% of the dataset). This testing 
data was used to evaluate the model and assess its accuracy, i.e. how well the 
model behaves with new data. 

Also, the implementation provides a testing source file named 
\texttt{test\_models.py} that serves as a POC\footnote{Proof-of-Concept} for 
the functionality of the project. Here, our implemented models will be 
automatically loaded and trained and then, compared to each other. The test module 
will provide an easy command line interface that will 
allow the user to give as input to the program some examples and then 
get insights about them. The user will provide the features for their examples 
and will choose the subset of features and also the model that will be used for 
prediction. 
