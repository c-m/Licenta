\chapter{Introduction}

Researchers in the last decades spent a lot of time developing tools and 
models for problems observed in nature and life. Their approach was purely 
analyitical and, given the fact that some systems behave in a very complex, 
non-linear way, these methods became harder and harder to build and 
understand. 

But, with the recent advances in computational power, another approach was 
used to solve those scientific problems. This approach is oriented on 
empirical observations, being driven by the data, rather than the model and 
is called Machine Learning. The main purpose of it is to learn complex patterns 
and relations in data, without providing the rules explicitly.

The educational environment is a good place to apply these automatic models, 
since a lot of data is present here, it is hard to manually make inferences 
about the given datasets and a general overview of the structure of data is 
almost impossible to obtain without the help of intelligent systems.

In this report we are going to use a Machine Learning approach to solve the 
problem of automatic grade prediction by using two popular models, i.e. 
Neural Networks and Random Forest. The workflow which our project will use 
for building the model is show in the next diagram: 

\insfigshw{diagram.png}%
    {Workflow of a ML model}%
    {Workflow of a ML model}%
    {fig.diagram}{0.8}


\section{Motivation}

Sometimes, evaluating the students performance in a course during the semester 
is found to be a difficult issue. The obstacles come from the fact that there 
are many students in an undergratuate class, and analyzing each of them is 
not so scalable, due to the limited number of course staff. Moreover, if the 
course is a difficult one and most students have problems passing it, we
want to help these students get on track before taking the final exam and 
failing it. 

For the problem stated above, a Machine Learning solution comes naturally, 
since it offers a way of classifying students and making predictions about 
their final grades. Also, with Machine Learning tools we can get insights 
about what grading components should we look at when we will want to assess 
the performance of a student before the course ending.

\section{Contributions}

The initial purpose of this project was to build an automated model for grading 
students based on the semester activity of each student. Later, this aim 
took the form of a model focused on getting important information from the 
data. Specifically, we wanted to use this model for helping students with 
a high probability of failing the course as early as possible during the semester. 

The contributions of the project include:

\begin{itemize}
\item A research done on Machine Learning techniques and software frameworks 
appropiate for our use case
\item A working model that proves the point of using Machine Learning for 
automatic grade predictions
\item An evaluation of the model, with focus on educational analysis
\item A generic way of using the developed model to other undergraduate courses 
related with the ones we used
\end{itemize}