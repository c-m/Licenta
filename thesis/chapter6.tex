\chapter{Conclusion}

In this thesis, we described the approaches taken to adding machine learning
models such as neural networks and random forest to automate the prediction of
grades for students in a Computer Science class. We analyzed in detail the
structure of our dataset and proposed a generic solution, that can be used  in
any Computer Science course. Then, we moved forward to discuss the chosen
models and their mathematical formulations. In the last part of the report, 
we commented on the implementation of the project and then, evaluated the 
implementation with various metrics.

We believed that we managed to solve the problems and goals for this thesis. 
After analyzing our data and getting the results of the experiments used, 
we infered some valuable information about the course grading methods. 

Then, choosing the best model for automatic learning is not a direct 
response. Neural networks can be slower but provide better results, whereas 
random forests are fast and give more information about the importance 
of features, but their predicting performance can be weaker.

We encountered some problems during the implementation of this project, 
but managed to find a proper solution with the little amount of data we got 
access to. There are also some ideas on improving this project, as pointed in 
the following section.

\section{Future Work}

Some next steps that we can take in the future with this work include: 

\begin{itemize}
\item Getting more data from different years and different courses to increase 
the number of training examples and, implicitly, the performance of the model
\item Implement an easy user interface for the model, so the course staff can 
benefit from this work
\item Finding more relevant features that can be used in the prediction. An 
example can be analyzing social interactions between students or finding 
patterns in their course attendance (lectures and seminars)
\end{itemize}
